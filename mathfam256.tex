%# -*- coding: utf-8 -*-
\ifx\epTeXinputencoding\undefined\else % defined in e-pTeX (> TL2016)
  \epTeXinputencoding utf8    % ensure utf-8 encoding for platex
\fi

\documentclass[a4paper]{jsarticle}
\usepackage{doc}
\usepackage{mathfam256}
\GetFileInfo{mathfam256.sty}
\usepackage{array}
\usepackage{url}
\newcommand{\OMEGA}{$\Omega$}
\def\pTeX{p\kern-.15em\TeX}
\def\upTeX{u\pTeX}
\def\eTeX{$\varepsilon$-\TeX}
\def\epTeX{$\varepsilon$-\pTeX}
\def\eupTeX{$\varepsilon$-\upTeX}
\title{Package \textsf{mathfam256} \fileversion}
\author{Hironobu Yamashita}
\date{\filedate}
\begin{document}

\maketitle

\parindent15pt

In classic \TeX\ and \eTeX, the upper limit of math groups (\verb+\fam+)
is limited to 16. However, these days some \TeX\ variants support
math groups up to 256.

In light of this situation, the \LaTeXe\ kernel 2015/01/01 increased
the upper limit of math alphabets to 256 for
Xe\TeX\ and Lua\TeX\ (Unicode \TeX\ variants).
Also, the \pLaTeXe/\upLaTeXe\ kernel 2016/11/29 introduced the similar
change for \epTeX/\eupTeX\ (``FAM256'' patch based on \OMEGA\ is applied).

However, the ``native'' kernel support is limited:
\begin{itemize}
\item Lamed (\LaTeX\ on Aleph) is not supported at all.
\item The upper limit for math symbol families is still limited to 16.
\end{itemize}

The package \textsf{mathfam256} is an attempt to resolve this situation;
increases the upper limit of math symbols from 16 (\LaTeXe\ default)
to 256, using \verb+\omath...+ primitives.

The package is maintained on GitHub:
\begin{quote}
  \url{https://github.com/aminophen/mathfam256}
\end{quote}

\section*{Requirements}

The \verb+\omath...+ primitives are originally introduced in
\OMEGA\ (Omega), and are currently available in the followings.
Without one of these, the package causes an error.
\begin{itemize}
\item \pLaTeX/\upLaTeX, run on \epTeX/\eupTeX\ (Japanese TeX engine)
\item Lamed, run on Aleph (successor of \OMEGA)
\end{itemize}

\section*{Usage}

Load it in the preamble; it is recommended to load it as fast as
possible.
\begin{verbatim}
  \documentclass[a4paper]{article}
  \usepackage{mathfam256}
  \usepackage{...}% other packages
  \begin{document}
  The content
  \end{document}
\end{verbatim}

\bigskip

\parindent1zw

オリジナルの\TeX や“素”の\eTeX では、数式ファミリ(\verb+\fam+)の
上限が16個に制限されていました。しかし、最近の新しい\TeX エンジンの中には、
256個までの数式ファミリをサポートするものが現れています。

この状況を鑑み、\LaTeXe カーネル(2015/01/01以降)では、
Unicode対応のXe\TeX とLua\LaTeX について、数式ファミリ(正確には
\verb+\DeclareMathAlphebet+で使われる数式アルファベット)の上限を256に
増やしています。同様に、\pLaTeXe/\upLaTeXe カーネル(2016/11/29以降)も、
\OMEGA ベースのいわゆる「FAM256パッチ」が適用された\epTeX/\eupTeX について、
数式ファミリの上限を256に増やしています。これらにより
\begin{verbatim}
  ! LaTeX Error: Too many math alphabets used in version normal.
\end{verbatim}
というエラーが発生しにくくなっています。

しかし、この“ネイティブな”カーネルのサポートには限界があります:
\begin{itemize}
\item Lamed (\LaTeX\ on Aleph)は全く考慮されていない。
\item 数式シンボルフォントの上限は依然として16のままである(安全のため)。
\end{itemize}

この\textsf{mathfam256}パッケージは、この状況を改善する試みです。
すなわち、\OMEGA 由来の\verb+\omath...+プリミティブを用いて、
数式シンボルフォントの上限を256に増やします。

\section*{動作条件}

もともと\OMEGA\ (Omega)に導入された\verb+\omath...+プリミティブが
必要です。これが実装されているのは、以下のフォーマット(エンジン)です。
これ以外でパッケージが使われて場合はエラー終了します。
\begin{itemize}
\item \pLaTeX/\upLaTeX (\epTeX/\eupTeX 上で動作)
\item Lamed(\OMEGA の後継であるAleph上で動作)
\end{itemize}

\section*{使い方}

プリアンブルで読み込みます。なるべく早めに読み込むことをお勧めします。
\begin{verbatim}
  \documentclass[a4paper]{article}
  \usepackage{mathfam256}
  \usepackage{...}% other packages
  \begin{document}
  …本文…
  \end{document}
\end{verbatim}

\section*{謝辞}

本パッケージは\cite{yamamoto}〜\cite{yamamoto3}を元に、
新しい\LaTeX に対応させることで作られました。

\begin{thebibliography}{99}
 \bibitem{yamamoto} 山本 和義,「数式famの制限とluatex」,
  掲示板「\TeX\ Q\ \&\ A」,2009/02/12.\\
  \url{http://oku.edu.mie-u.ac.jp/~okumura/texfaq/qa/52744.html}
 \bibitem{yamamoto2} 山本 和義,「Re: 数式famの制限とluatex」,
  掲示板「\TeX\ Q\ \&\ A」,2009/02/16.\\
  \url{http://oku.edu.mie-u.ac.jp/~okumura/texfaq/qa/52767.html}
 \bibitem{yamamoto3} 山本 和義,「数式fam拡張マクロ for e-pTeX等」,
  掲示板「\TeX\ Q\ \&\ A」,2009/02/21.\\
  \url{http://oku.edu.mie-u.ac.jp/~okumura/texfaq/qa/52799.html}
 \bibitem{eptexdoc} 北川 弘典,「\epTeX について」.\\
  \verb+$TEXMFDIST/doc/ptex/ptex-base/eptexdoc.pdf+
\end{thebibliography}

\end{document}
